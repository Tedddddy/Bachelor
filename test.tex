\documentclass[12pt, a4paper]{scrreprt}
\usepackage{enumitem}
\usepackage{stmaryrd} 
\usepackage{amsfonts}
\usepackage{amsthm}
\begin{document}

\chapter{Definition}

\section{Defintion }
Let Prop be a set of variables. Then a formula $\phi$ is defined as follows :
$$\phi ::= p \mid \bot \mid \phi \mid \phi \rightarrow \phi \mid \Box_i \phi$$
where $p \in Prop$ and $\Box_i$ is a modal operator. Other connectives are expressed through $\bot$ and $\rightarrow$ and 
dual modal operators $\diamond_i$ as $\diamond_i \phi = \neg \Box_i \neg \phi$

\section{Defintion }
A normal modal logic is a set of modal formulas containing all propositional tautologies,
closed under Substitution ($\frac{\phi(p_i)}{\phi(\psi)}$), Modus Ponens 
$(\frac{\phi, \phi \rightarrow \psi}{\psi})$, Generalization rules $(\frac{\phi}{\Box_i \phi})$
and the following axioms 
$$ \Box_i (p \rightarrow q) \rightarrow (\Box_i p \rightarrow \Box_i q)$$

$K_n$ denotes the minimal normal modal logic with n modalities and $K = K_1$
Let L be a logic and let $\Gamma$ be a set of formulas. Then L+$\Gamma$ denotes 
the minimal logic containing L and $\Gamma$

\section{Definition}
Let L1 and L2 be two modal logic with one modality $\Box$. Then the fusion of these 
logics are defined as follows :
$$ L1 \otimes L2 = K2 + L_{1(\Box \rightarrow \Box_1)} L_{2(\Box \rightarrow \Box_2)} $$

The follow logics may be important 

$$D = K + \Box p \rightarrow \Diamond p$$
$$T = K + \Box p \rightarrow p$$
$$D4 = D + \Box p \rightarrow \Box \Box p$$
$$S4 = T + \Box p \rightarrow \Box \Box p$$


\chapter{Topological Space Defintion}

\section{Defintion}

A topological space is a pair $(X, \tau)$ where $\tau$ is a collection of subsets of X (elements of $\tau$ are also called open sets) such that : 
\newline
\newline
1. the empty set $\emptyset $ and X are open
\newline
2. the union of an arbitrary collection of open sets is open
\newline
3. the intersection of finite collection of open sets is open
\newline
\newline
A topological model is a structure M = (X,$\tau$,v) where (X,$\tau$) is a topological space
and v is a valuation assigining subsets of X to propositional variables.

\section{Defintion}
Let M = (X,$\tau$,v) a topological model and $x \in X$. The satisfaction of a formula
at the point x in M is defined inductively as follows :$M,x \models \Box \phi$ iff $,\exists U \in \tau$ s.t $x \in U$ and $\forall u \in U : M,u \models \phi$
\newline
$M,x \models \Box \phi$ iff $,\exists U \in \tau$ s.t $x \in U$ and $\forall u \in U : M,u \models \phi$
\newline
$M,x \models \Diamond \phi$ iff $,\forall U \in \tau$ s.t $x \in U$ and $\exists u \in U : M,u \models \phi$

\section{Defintion}
Let $A = (X, \chi)$ and B =(Y, $\upsilon$) be topological spaces. The standard product topology $\tau$ is the set of subsets of 
$X \times Y$ such that $X \in \chi$ and $Y \in \upsilon$. \newline
Let $N \subseteq X \times Y $. We call $N$ horizontally open if $\forall (x,y) \in N $ $\exists U \in \chi : x \in U $ and $ U \times \{ y \} \subseteq N$. \newline We call $N$ 
vertically open if $\forall (x,y) \in N$ $\exists V \in \upsilon : y \in V$ and  $ \{ x \} \times V \subseteq N$ \newline
If $N$ is H-open and V-open, then we call it HV-open. \newline
We denote $\tau_1$ is the set of all H-open subsets of $X \times Y$ and $\tau_2$ is the set of all V-open subsets of $X\times Y$


\chapter{Neighbourhood} 

\section{Defintion} 
Let X be a non-empty set. A function  $\tau : X \rightarrow 2^{2^X}$ is called a neighbourhood function. A pair 
F = (X,$\tau$) is called a neighbourhood frame (or n-frame). A model based on F is a tuple (X,$\tau$,v), where v assigns a subset of X to a variable

\section{Defintion}

Let $M$ =(X,$\tau$,v) be a neighourhood model and x $\in$ X. The truth of a formula is defined inductively as follows :
$$M,x \models \Box \phi \mbox{ iff } \exists V \in N(x) \forall y \in V : M,y \models \phi$$ 
A formula is valid in a n-model M if it is valid at all points of $M$ ($M \models \phi$). Formula is valid in a n-frame $F$ if it is valid in
all models based on $F$ (notation $F \models \phi$). For Logic L we write $ F \models L, \mbox{ if for any }\phi \in L, F \models \phi$. 
$\mbox{We define nV(L) =  } \{ F \mid F \mbox{ is an n-frame and } F \models \phi \}$.

\section{Defintion}
Let F = (W,R) be a Kripke frame. We define an n-frame $N$(F) = (W, $\tau$) as follows.
For any $w\in W$ we have :
$$\tau(w) = \{ U \mid R(w) \subseteq U \subseteq W \}$$

\section{Defintion}
Let $X$ = (X, $\tau_1$,...) and $Y$ = (Y, $\sigma_1$,...) be n-frames. Then the function f:
$X \rightarrow Y$ is called bounded morphism if \newline \newline
1. f is surjective \newline
2. $\forall x\in X \, \forall U \in \tau_i(x) : f(U) \in \sigma_i (f(x))$ \newline
3. $\forall x\in X \, \forall V \in \sigma_i (f(x)) \, \exists U \in \tau_i(x) \, : f(U) \subseteq V$

\section {Defintion}












\end{document}