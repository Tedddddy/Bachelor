\documentclass[12pt, a4paper]{scrreprt}
\usepackage{enumitem}
\usepackage{stmaryrd} 
\usepackage{amsfonts}
\usepackage{amsthm}
\usepackage{amssymb}
\begin{document}

\chapter{Definition}

\section{Defintion }
Let prop be a set of variables. Then a formula $\phi$ is defined as follows :
$$\phi ::= p \mid \bot \mid \phi \mid \phi \rightarrow \phi \mid \Box_i \phi$$
where $p \in Prop$ and $\Box_i$ is a modal operator. Other connectives are expressed through $\bot$ and $\rightarrow$ and 
dual modal operators $\diamond_i$ as $\diamond_i \phi = \neg \Box_i \neg \phi$

\section{Defintion }
A normal modal logic is a set of modal formulas containing all propositional tautologies,
closed under Substitution ($\frac{\phi(p_i)}{\phi(\psi)}$), Modus Ponens 
$(\frac{\phi, \phi \rightarrow \psi}{\psi})$, Generalization rules $(\frac{\phi}{\Box_i \phi})$
and the following axioms 
$$ \Box_i (p \rightarrow q) \rightarrow (\Box_i p \rightarrow \Box_i q)$$

$K_n$ denotes the minimal normal modal logic with n modalities and $K = K_1$
Let L be a logic and let $\Gamma$ be a set of formulas. Then L+$\Gamma$ denotes 
the minimal logic containing L and $\Gamma$

\section{Definition}
Let L1 and L2 be two modal logic with one modality $\Box$. Then the fusion of these 
logics are defined as follows :
$$ L1 \otimes L2 = K2 + L_{1(\Box \rightarrow \Box_1)} L_{2(\Box \rightarrow \Box_2)} $$

The follow logics may be important 

$$D = K + \Box p \rightarrow \Diamond p$$
$$T = K + \Box p \rightarrow p$$
$$D4 = D + \Box p \rightarrow \Box \Box p$$
$$S4 = T + \Box p \rightarrow \Box \Box p$$

\chapter{Kripke Definition}
To show D $\otimes$ D $\otimes$ D + $\Box p \rightarrow \Box_1 p \land \Box_2 p$ has the finite model property, we need some important defintions.

\section{Defintion}

Let $M = (W,R,V)$ be a model and $w \in W$ a state in $M$. The notion of a formula being true at w is inductively defined as follows :
$$M, w \Vdash \Box \phi \mbox{ iff } \forall v \in W : wRv \rightarrow M,v \Vdash \phi$$
\newline
This defintion can be extended to a multimodal version, where the modal operators are interpreted the same way but with the respective relation.

\section{Defintion}
A set $\Sigma$ is closed under subformulas, if for all formulas $\phi$ and $\phi'$ the following holds :
\newline \newline
1. if $\neg \phi \in \Sigma$ then $\phi \in \Sigma$
\newline
2. if $\phi \lor \phi' \in \Sigma$ then $\phi, \phi' \in \Sigma$
\newline
3. if $\Box \phi \in \Sigma$ then $\phi \in \Sigma$
\newline
\newline
We can define it similary for a multimodal logic. For every modal operator $\Box_n$, we extend this definition by adding a new condition similar to the third.
Example : Suppose we have a multimodal logic with $\Box, \Box_1$ called $L_{\Box, \Box_1}$ and $\phi = \Box p \rightarrow \Box_1 q$ and $\phi \in \Sigma$. Then $\Sigma = \{ \phi, \neg\Box p, \Box_1 q, \Box p, p, q\}$ is closed under
subformulas.

\section{Defintion}
Let $M = (W,R,V)$ be a model and suppose $\Sigma$ is a set of formulas. We define a relation $\equiv$ on W as follows : \newline 
$$w \equiv v \mbox{ iff } \forall \phi \in \Sigma : M,w \Vdash \phi \Leftrightarrow M,v \Vdash \phi$$ 
It is well known that the $\equiv$-relation is an equivalence relation. We denote the equivalence class of a state $w\in W$ as $[w]_\Sigma = \{v \mid v \equiv w\}$. Furthermore $W_\Sigma$ is the set of all equivalence classes, i.e
$W_\Sigma = \{[w]_\Sigma \mid w \in W\}$.

\section{Defintion}

Let $M = (W,R,V)$ be a model, $\Sigma$ is closed under subformulas and $W_\Sigma$ the set of equivalence classes induced by $\equiv$.
A model $M^f_\Sigma = (W^f, R^f, V^f)$ is called filtration of M through $\Sigma$ if the following holds : \newline \newline
1. $W^f = W_\Sigma$ \newline
2. If $(w,v) \in R$ then $([w],[v]) \in R^f$ \newline
3. If $([w], [v]) \in R^f$ then for any $\Box \phi \in \Sigma$ : if $M,w \Vdash \Box \phi$ then $M,v \Vdash \phi$  \newline 
4. $V^f = \{[w] \mid M,w \Vdash p\}$, for all propositional variables $p \in \Sigma$  
\newline \newline
In our case, we are in a multimodal logic with three modal operators $\Box, \Box_1,\Box_2$. We need to extend this defintion for  $L_{\Box,\Box_1,\Box_2}$. This means our model looks like:
$M = (W, R,R_1, R_2,V)$. We extend the conditions as follows : \newline \newline
If $ (w,v) \in R_i$ then $([w], [v]) \in R^f_i$ \newline
If $ ([w], [v]) \in R^f_i$ then for any $\Box_i \phi \in \Sigma : \mbox{ if } M,w \Vdash \Box_i$ then $M,v \Vdash \phi$ ,  where $i \in \{1,2\}$

\section{Filtration Theorem}
Consider $L_{\Box, \Box_1, \Box_2}$. Let $M^f =(W_\Sigma, R^f, R^f_1, R^f_2, V)$ be a filtration of M through a subformula closed set $\Sigma$.
Then for all formulas $\phi \in \Sigma$, and all nodes $w \in M$, we have $$M,w \Vdash \phi \mbox{ iff } M^f,[w] \Vdash \phi$$

\begin{proof}
By induction on $\phi$. We will only show non-trivial and, for our purposes, necessary cases. \newline

Case $\phi = p$: Left to right follows immediately from filtration defintion. Conversely, suppose
$M^f, [w] \Vdash p$. This means $ [w] \in V^f(p)$. But this means V(p) can not be empty. Pick any $v\in V(p)$. Obviously, $w\equiv v$ and
$M,v\Vdash p$. Hence, $M,w \Vdash p$. \newline

Case $\phi = \neg \psi$: Suppose $\psi$ holds. Then we have : $M,w \Vdash \phi \mbox{ iff } M,w \nvDash \psi$. Applying 
induction hypothesis, we get : $M^f, [w] \nvDash \psi$. But then, we have $M^f, [w] \Vdash \phi$. Right to left is the same. \newline

Case $\phi = \phi_1 \land \phi_2$: Suppose $\phi_1, \phi_2$ holds. Let $M,w \Vdash \phi$. That means $M,w \Vdash \phi_1$ and $M,w \Vdash \phi_2$. Applying
induction hypothesis, we get $M^f,[w] \Vdash \phi_1$ and $M^f,[w] \Vdash \phi_2$. But then, $M^f,[w] \Vdash \phi_1 \land \phi_2 = \phi$. Right to left is similar. \newline

Case $\phi = \Box_i\psi$ ($i \in \{1,2,\epsilon\}, \Box_\epsilon = \Box$): Left to right. Suppose $\psi$ holds and  $M,w \Vdash \Box_i \psi$. 
We need to show $M^f,[w] \Vdash \Box_i \psi$, this means $\forall [v] \in W_\Sigma : [w]R_i[v] \rightarrow M^f,[v] \Vdash \Box_i \psi$. Pick any $[v] \in W_\Sigma$ s.t 
$[w]R_i[v]$. By condition 3, w.r.t to the modal operator, we have $M,v \Vdash \psi$. By induction hypothesis, we get $M^f,[v] \Vdash \psi$. Because [v] was arbitrary it follows that $M^f,[w] \Vdash \Box_i \psi$. \newline \newline
Right to left. Suppose $\psi$ holds and $M^f,[w] \Vdash \Box_i \psi$. Pick $v \in W$ s.t $wR_iv$. By condition 2, w.r.t to the modal operator, we have
$[w]R^f_i[v]$. So, $M^f, [v] \Vdash \psi$. By induction hypothesis, we get $M,v \Vdash \psi$. Because v was arbitrary, we have $M,w \Vdash \Box_i \psi$.

\end{proof} 

Now define the smallest filter for $L_{\Box,\Box_1,\Box_2}$ and show that this is a filter. We denote this as $R^s$.

\section{Defintion}
 Let $M = (W,R,V)$ be a model, $\Sigma$ is closed under subformulas and $W_\Sigma$ the set of equivalence classes. We define :
$$R^s = {[w],[v] \mid }$$





\chapter{Topological Space Defintion}

\section{Defintion}

A topological space is a pair $(X, \tau)$ where $\tau$ is a collection of subsets of X (elements of $\tau$ are also called open sets) such that : 
\newline
\newline
1. the empty set $\emptyset $ and X are open
\newline
2. the union of an arbitrary collection of open sets is open
\newline
3. the intersection of finite collection of open sets is open
\newline
\newline
The space is called Alexandroff, if we allow the intersection of infinite collection of open sets.
A topological model is a structure M = (X,$\tau$,v) where (X,$\tau$) is a topological space
and v is a valuation assigining subsets of X to propositional variables. 

\section{Defintion}
Let M = (X,$\tau$,v) a topological model and $x \in X$. The satisfaction of a formula
at the point x in M is defined inductively as follows :
\newline
$M,x \models \Box \phi$ iff $\, \exists U \in \tau$ s.t $x \in U$ and $\forall u \in U : M,u \models \phi$
\newline
$M,x \models \Diamond \phi$ iff $\, \forall U \in \tau$ s.t $x \in U$ and $\exists u \in U : M,u \models \phi$

\section{Defintion}
Let $A = (X, \chi)$ and B =(Y, $\upsilon$) be topological spaces. The standard product topology $\tau$ is the set of subsets of 
$X \times Y$ such that $X \in \chi$ and $Y \in \upsilon$. \newline
Let $N \subseteq X \times Y $. We call $N$ horizontally open if $\forall (x,y) \in N $ $\exists U \in \chi : x \in U $ and $ U \times \{ y \} \subseteq N$. \newline We call $N$ 
vertically open if $\forall (x,y) \in N$ $\exists V \in \upsilon : y \in V$ and  $ \{ x \} \times V \subseteq N$ \newline
If $N$ is H-open and V-open, then we call it HV-open. \newline
We denote $\tau_1$ is the set of all H-open subsets of $X \times Y$ and $\tau_2$ is the set of all V-open subsets of $X\times Y$

\section{Defintion}
Let X and Y be topological spaces and $f : X \rightarrow Y$ a function.
We call f continuous if for each open set $U \subseteq Y$ the set $f^{-1}(U)$ is open in X. We say f 
is open if for each open set $V \subseteq X$ the set f[V] is open in Y.

\section{Remark}
There is an alternative defintion for open sets. Let (X,$\tau$) be a topological space and U a set.
U is open iff $\forall x \in U$  $\exists V\subseteq U$ : V is open and $x \in V$. This is true because,
the union of open sets is an open set.
\newline
\newline

Now we define some Kripke frames, which we will use through this chapter.

\section{Defintion}
Let $T_2$ be the infinite binary tree with reflexive and transitive descendant relation. \newline Formally it is defined as follows :
$T_2 = (W,R)$ where W = \{0,1\}* and sRt iff $\exists u\in W : s*u = t$. \newline
The $T_ {6,2,2}$ tree is the infinite six branching tree, where all nodes of $T_ {6,2,2}$ is R-related, the first two R1-related 
and the last two R2-related. Formally we can define this tree as follows : $T_ {6,2,2} = (W,R,R_1,R_2)$,
where W = \{{0,1,2,3,4,5}\}*, $$sRt \mbox{ iff } \exists u \in \mbox{\{0,1,2,3,4,5\}*} : s*t = u $$
$$ sR_1t  \mbox{ iff } \exists u \in \mbox{\{0,1\}*} : s*t = u$$
$$ sR_2t  \mbox{ iff } \exists u \in \mbox{\{5,6\}*} : s*t = u$$

where s and t are elements of the set where the element u can come from, w.r.t to the relation. For example in the case sRt,
s and t are elements of \{0,1,2,3,4,5\}*.





\chapter{Neighbourhood} 

\section{Defintion} 
Let X be a non-empty set. A function  $\tau : X \rightarrow 2^{2^X}$ is called a neighbourhood function. A pair 
F = (X,$\tau$) is called a neighbourhood frame (or n-frame). A model based on F is a tuple (X,$\tau$,v), where v assigns a subset of X to a variable

\section{Defintion}

Let $M$ =(X,$\tau$,v) be a neighourhood model and x $\in$ X. The truth of a formula is defined inductively as follows :
$$M,x \models \Box \phi \mbox{ iff } \exists V \in N(x) \forall y \in V : M,y \models \phi$$ 
A formula is valid in a n-model M if it is valid at all points of $M$ ($M \models \phi$). Formula is valid in a n-frame $F$ if it is valid in
all models based on $F$ (notation $F \models \phi$). For Logic L we write $ F \models L, \mbox{ if for any }\phi \in L, F \models \phi$. 
$\mbox{We define nV(L) =  } \{ F \mid F \mbox{ is an n-frame and } F \models \phi \}$.

\section{Defintion}
Let F = (W,R) be a Kripke frame. We define an n-frame $N$(F) = (W, $\tau$) as follows.
For any $w\in W$ we have :
$$\tau(w) = \{ U \mid R(w) \subseteq U \subseteq W \}$$

\section{Defintion}
Let $X$ = (X, $\tau_1$,...) and $Y$ = (Y, $\sigma_1$,...) be n-frames. Then the function f:
$X \rightarrow Y$ is called bounded morphism if \newline \newline
1. f is surjective \newline
2. $\forall x\in X \, \forall U \in \tau_i(x) : f(U) \in \sigma_i (f(x))$ \newline
3. $\forall x\in X \, \forall V \in \sigma_i (f(x)) \, \exists U \in \tau_i(x) \, : f(U) \subseteq V$

\section {Defintion}
Let $X$ = (X, $\tau_1$) and $Y$ = (Y, $\tau_2$) be two n-frames. Then the product of these two frames
is an n-2-frame and is defined as follows : \newline

$$ X \times Y = (\mbox{X} \times \mbox{Y}, \tau_1', \tau_2')$$   
$$ \tau_1'(x,y) = \{ U \subseteq \mbox{X} \times \mbox{Y} \mid \exists V \in \tau_1(x) : V \times  \{ y \} \subseteq U \}$$
$$ \tau_2'(x,y) = \{ U \subseteq \mbox{X} \times \mbox{Y} \mid \exists V \in \tau_2(y) : \{ x \} \times V \subseteq U \}$$

\section {Defintion}

For two unimodal logics $L_1$ and $L_2$ we define the n-product of them as follows :
$$ L_1 \times_n L_2 = Log(\{ X \times Y \mid X \in nV(L_1) \mbox{ and } Y \in nV(L_2) \})$$

Now we define some Kripke frames we need for this chapter.


\section{Defintion}
Let $T_{\omega [in]}$ (i = irreflexiv, n = non-transitiv) denote the infinite branching and infinite depth tree, which is irreflexiv and non-transitive.
Formally the tree can be defined as : $T_{\omega [in]} = (W, R)$ where $W = \mathbb{N}$* and sRt iff $\exists u \in \mathbb{N} : s*u = t$ (the '*' is the concatenation operator) \newline \newline
The $T_{\omega,\omega,\omega [in]}$ tree is similary defined as the $T_{6,2,2}$ tree but with infinite branching and infinite depth. Before characterizing it, we say
$\mathbb{N}_{R1}$* is the set of finite number combinations which has a subscript $R_1$ to denote that these numbers relate to $R_1$  \newline 
(examples : $0_{R1}, 0123_{R1}$).
$\mathbb{N}^+_{R_1}$ is the set $\mathbb{N}_{R_1}$*- \{$\epsilon$\}. $\mathbb{N}^+_{R}$, $\mathbb{N}^+_{R_2}$ are defined similar. \newline \newline
Now let $T_{\omega,\omega,\omega [in]} = (W, R, R_1, R_2 )$ where $W = \mathbb{N}^+_{R} \cup \mathbb{N}^+_{R_1} \cup \mathbb{N}^+_{R_2} \cup \{ \epsilon\}$, 
$$sRt \mbox{ iff } \exists u \in \mathbb{N}_{R} \cup \mathbb{N}_{R_1} \cup \mathbb{N}_{R_2} : s*u = t$$
$$sR_1t \mbox { iff } \exists u \in \mathbb{N}_{R_1} : s * u = t$$
$$sR_2t \mbox { iff } \exists u \in \mathbb{N}_{R_2} : s * u = t$$
where s,t are elements of the positive closure set where the element u can come from (additionally s can be $\epsilon$), w.r.t to the relation. For example
if we consider $sR_1t$, then $s,t \in \mathbb{N}^+_{R_1}$ but also $s = \epsilon$. Of course the '*' operator acts here again as a concatenation operator.

















\end{document}

