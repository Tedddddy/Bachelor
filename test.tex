\documentclass[12pt, a4paper]{scrartcl}
\usepackage{enumitem}
\usepackage{stmaryrd} 
\usepackage{amsfonts}
\usepackage{amsthm}
\usepackage{amssymb}
\usepackage{amsmath}
\usepackage{hyperref}

\newtheorem{definition}{Definition}[subsection]
\newtheorem{lemma}[definition]{Lemma}
\newtheorem{proposition}[definition]{Proposition}
\newtheorem{corollary}[definition]{Corollary}
\newtheorem{theorem}[definition]{Theorem}
\newtheorem{remark}[definition]{Remark}


\begin{document}
\section{Introduction}
The study of topological semantics in modal logic was initiated by McKinsey and Tarski in 1944 [?]. The idea was to generalize Kripke frames using tools from topology.
 Neighbourhood semantics [?] as a generalization of Kripke semantics for modal logic were invented independently by Dana Scott [?] and Richard Montague [?].
Neighborhood semantics is more general than Kripke semantics and in the case of normal reflexive and transitive logics coincides with topological semantics.
The original motivation for introducing was to provide a semantics for non-normal modal logics. But in
recent years, interest in topological semantics and neighborhood frames has grown considerably, partly due to its applications in artificial intelligence. \newline
Oftentimes, it is necessary to combine frames for different modal logics into a complex
frame. The natural way of doing that is a product construction. For Kripke frames,
the resulting product is the Cartesian product of the two frames with two accessibility
relations. For topological semantics, the product of topogical spaces as bi-topological
spaces with so-called horizontal and vertical topologies have been considered. In a similar
fashion, the product of neighborhood frames was introduced by Sano in [?]. \newline
Now, let $L_1$ and $L_2$ be two modal logics. We say $L_1\otimes L_2$ (called fusion) is the 
minimal modal logic containing $L_1$ and $L'_2$, where $L'_2$ is the logic $L_2$ after renaming all modalities.
Furthermore, we say $L_1 \times_n L_2$ is the logic (i.e the set of all valid formulas) of the class 
products of neighbourhood frames $N_1 \times_n N_2$ such that $L_i$ is valid in $N_i$ for i = 1,2.
It was proven in [?] that for any two logics $L_1, L_2 \in \{D4,D,T,S4\}: $  $L_1\times_n L_2 = L_1 \otimes L_2$.
In [?], the authors studied a product of two spaces with three topologies : horizontal, vertical and
classic product topology. They proved that the logic of such spaces is $S4 \otimes S4 \otimes S4 + \Box p \rightarrow \Box_1 p \land \Box_2 p$
where $\Box$ corresponds to the product topology. \newline
The following work will present a detailed proof of the shown results. Additionally, we will show that $T\otimes T\otimes T + \Box p \rightarrow \Box_1 p \land \Box_2p$ 
 (we abbreviate it with TNL) = $T \,x^+_n \, T$. The proof ideas here are inspired by the shown results.
 \clearpage



 \section{Preliminaries}
 

\begin{definition}
Let prop be a set of variables. Then a formula $\phi$ is defined as follows:
$$\phi ::= p \mid \bot \mid \phi \mid \phi \rightarrow \phi \mid \Box_i \phi$$
where $p \in Prop$ and $\Box_i$ is a modal operator. Other connectives are expressed through $\bot$ and $\rightarrow$ and 
dual modal operators $\diamond_i$ as $\diamond_i \phi = \neg \Box_i \neg \phi$
\end{definition}

\begin{definition}
A normal modal logic is a set of modal formulas containing all propositional tautologies,
closed under Substitution ($\frac{\phi(p_i)}{\phi(\psi)}$), Modus Ponens 
$(\frac{\phi, \phi \rightarrow \psi}{\psi})$, Generalization rules $(\frac{\phi}{\Box_i \phi})$
and the following axioms 
$$ \Box_i (p \rightarrow q) \rightarrow (\Box_i p \rightarrow \Box_i q)$$

$K_n$ denotes the minimal normal modal logic with n modalities and $K = K_1$
Let L be a logic and let $\Gamma$ be a set of formulas. Then L+$\Gamma$ denotes 
the minimal logic containing L and $\Gamma$
\end{definition}


\begin{definition}
Let $L_1$ and $L_2$ be two modal logic with one modality $\Box$. Then the fusion of these 
logics are defined as follows :
$$ L_1 \otimes L_2 = K_2 + L_{1(\Box \rightarrow \Box_1)} L_{2(\Box \rightarrow \Box_2)} $$

The following logics may be important 

$$D = K + \Box p \rightarrow \Diamond p$$
$$T = K + \Box p \rightarrow p$$
$$D4 = D + \Box p \rightarrow \Box \Box p$$
$$S4 = T + \Box p \rightarrow \Box \Box p$$   
\end{definition}

Now we introduce some sepcial kind of frames, which we will use through this work.
\begin{definition}
    Let A be a nonempty set.
    $$A^* = \{a_1...a_k \mid a_i \in A\}$$ 
    is the set of all finite sequences of elements from A, including the empty sequence $\Lambda$.
    Elements from $A^*$ will be denoted as $\vec{a}$. The length of a sequence $\vec{a} = a_1...a_k$ is k (also $l(\vec{a}) = k$) 
    and the length of $\Lambda$ is $0$ $(l(\Lambda) = 0)$. Concatenation is denoted by $"\cdot"$ : $(a_1...a_k) \cdot (b_1...b_l) = \vec{a} \cdot \vec{b} = a_1...a_kb_1...b_l$.

\end{definition}

\begin{definition}
    Let A be a nonempty set. We define an infinite frame $F_{in}[A] = (A^*, R)$ s.t for $\vec{a}, \vec{b} \in A^*$ 
    $$\vec{a}R\vec{b} \Leftrightarrow \exists x \in A \, (\vec{b} = \vec{a} \cdot x)$$
    Furthermore we define : 
    $$F_{rn}[A] = (A^*,R^r), \mbox{ where } R^r = R \cup Id \mbox{ (reflexive closure)}$$
    $$F_{it}[A] = (A^*,R^*), \mbox{ where } R^* = \bigcup_{i=1}^{\infty}R^i \mbox{ (transitive closure)}$$
    $$F_{rt}[A] = (A^*, R^{r*})$$ \newline
    where "$t$" stands for transtive, "$n$" for non-transitive, "$r$" for reflexive and "$i$" for irreflexive. \newline
    For now, we will use the following notion to generalize : $F_{\xi \eta}$ where $\xi \in \{i,r\}$ and $\eta \in \{t,n\}$ \newline
\end{definition}

\begin{proposition}
    Let $F =F_{\xi \eta}[A] = (A^*, R)$ then $$\vec{a}R(\vec{a} \cdot \vec{c}) \Leftrightarrow \Lambda R \vec{c}$$ 
\end{proposition}

\begin{definition}
    Let $F_1 = F_{\xi_1 \eta_1}[A] = (A^*,R_1)$ and $F_2 = F_{\xi_2 \eta_2}[B] = (B^*,R_2)$, where $\xi_1, \xi_2 \in \{i,r\}$ 
    and $\eta_1, \eta_2 \in \{t,n\}$. Furthermore, we assume $A = \{a_1, a_2,...\}$ and $B = \{b_1,b_2,...\}$ with $A \cap B = \emptyset$. Then we define the frame $F_1 \otimes F_2 = (W, R'_1, R'_2)$ as follows :
    $$W = (A \cup B)^*$$
    $$\vec{x} R'_1 \vec{y} \Leftrightarrow \vec{y} = \vec{x} \cdot \vec{z} \mbox{ for some } \vec{z} \in A^* \mbox{such that }\Lambda R_1 \vec{z}$$
    $$\vec{x} R'_2 \vec{y} \Leftrightarrow \vec{y} = \vec{x} \cdot \vec{z} \mbox{ for some } \vec{z} \in B^* \mbox{such that }\Lambda R_2 \vec{z}$$ \newline
\end{definition}

\begin{proposition} [\cite{ref6}, \cite{ref4}]
    Let $F_1$ and $F_2$ be as in Defintion 2.0.7. Then 
    $$Log(F_1 \otimes F_2) = Log(F_1) \otimes Log(F_2)$$
\end{proposition}

\begin{proposition}
    Let $F_{in} = F_{in}[\mathbb{N}], F_{rn} = F_{rn}[\mathbb{N}], F_{it} = F_{it}[\mathbb{N}] \mbox{ and } F_{rt} = F_{rt}[\mathbb{N}]$. Then the following holds: \newline 
    $$ Log(F_{in}) = D$$
    $$ Log(F_{rn}) = T$$
    $$ Log(F_{it}) = D4$$
    $$ Log(F_{rt}) = S4$$

\end{proposition}






\section{Completeness result for TNL}

We introduce a frame called $T_{\omega, \omega, \omega[rn]}$. We will show that TNL is sound and complete w.r.t $T_{\omega, \omega, \omega[in]}$
The idea is to pick a class of frame $C$ s.t $Log(C) = Log(T_{\omega, \omega, \omega[rn]})$ and then show that the class has FMP.
In the end, we will use an unravelling technique to show completeness. 

\subsection{Filtration and Sahlqvist for multimodal logic}

\begin{definition}
    Let $M = (W,R,V)$ be a model and $w \in W$ a state in $M$. The notion of a formula being true at w is inductively defined as follows :
    $$M, w \Vdash \Box \phi \mbox{ iff } \forall v \in W : wRv \rightarrow M,v \Vdash \phi$$
    \newline
    This defintion can be extended to a multimodal version, where the modal operators are interpreted the same way but with the respective relation.   
\end{definition}

\begin{definition}
    A set $\Sigma$ is closed under subformulas, if for all formulas $\phi$ and $\phi'$ the following holds :
    \newline \newline
    1. if $\neg \phi \in \Sigma$ then $\phi \in \Sigma$
    \newline
    2. if $\phi \lor \phi' \in \Sigma$ then $\phi, \phi' \in \Sigma$
    \newline
    3. if $\Box \phi \in \Sigma$ then $\phi \in \Sigma$
    \newline
    \newline
    We can define it similary for a multimodal logic. For every modal operator $\Box_n$, we extend this definition by adding a new condition similar to the third.
    Example : Suppose we have a multimodal logic with $\Box, \Box_1$ called $L_{\Box, \Box_1}$ and $\phi = \Box p \rightarrow \Box_1 q$ and $\phi \in \Sigma$. Then $\Sigma = \{ \phi, \neg\Box p, \Box_1 q, \Box p, p, q\}$ is closed under
    subformulas.
\end{definition}

\begin{definition}
    Let $M = (W,R,V)$ be a model and suppose $\Sigma$ is a set of formulas. We define a relation $\equiv$ on W as follows : \newline 
    $$w \equiv v \mbox{ iff } \forall \phi \in \Sigma : M,w \Vdash \phi \Leftrightarrow M,v \Vdash \phi$$ 
    It is well known that the $\equiv$-relation is an equivalence relation. We denote the equivalence class of a state $w\in W$ as $[w]_\Sigma = \{v \mid v \equiv w\}$. Furthermore $W_\Sigma$ is the set of all equivalence classes, i.e
    $W_\Sigma = \{[w]_\Sigma \mid w \in W\}$.    
\end{definition}

\begin{definition}
    Let $M = (W,R,V)$ be a model, $\Sigma$ is closed under subformulas and $W_\Sigma$ the set of equivalence classes induced by $\equiv$.
    A model $M^f_\Sigma = (W^f, R^f, V^f)$ is called filtration of M through $\Sigma$ if the following holds : \newline \newline
    1. $W^f = W_\Sigma$ \newline
    2. If $(w,v) \in R$ then $([w],[v]) \in R^f$ \newline
    3. If $([w], [v]) \in R^f$ then for any $\Box \phi \in \Sigma$ : if $M,w \Vdash \Box \phi$ then $M,v \Vdash \phi$  \newline 
    4. $V^f = \{[w] \mid M,w \Vdash p\}$, for all propositional variables $p \in \Sigma$  
    \newline \newline
    In our case, we are in a multimodal logic with three modal operators $\Box, \Box_1,\Box_2$. We need to extend this defintion for  $L_{\Box,\Box_1,\Box_2}$. This means our model looks like:
    $M = (W, R,R_1, R_2,V)$. We extend the conditions as follows : \newline \newline
    If $ (w,v) \in R_i$ then $([w], [v]) \in R^f_i$ \newline
    If $ ([w], [v]) \in R^f_i$ then for any $\Box_i \phi \in \Sigma : \mbox{ if } M,w \Vdash \Box_i$ then $M,v \Vdash \phi$ ,  where $i \in \{1,2\}$    
\end{definition}

\begin{theorem}
    Consider $L_{\Box, \Box_1, \Box_2}$. Let $M^f =(W_\Sigma, R^f, R^f_1, R^f_2, V)$ be a filtration of M through a subformula closed set $\Sigma$.
    Then for all formulas $\phi \in \Sigma$, and all nodes $w \in M$, we have $$M,w \Vdash \phi \mbox{ iff } M^f,[w] \Vdash \phi$$
    
    \begin{proof}
        By induction on $\phi$. We will only show non-trivial and, for our purposes, necessary cases. \newline
        
        Case $\phi = p$: Left to right follows immediately from filtration defintion. Conversely, suppose
        $M^f, [w] \Vdash p$. This means $ [w] \in V^f(p)$. But this means V(p) can not be empty. Pick any $v\in V(p)$. Obviously, $w\equiv v$ and
        $M,v\Vdash p$. Hence, $M,w \Vdash p$. \newline
        
        Case $\phi = \neg \psi$: Suppose $\psi$ holds. Then we have : $M,w \Vdash \phi \mbox{ iff } M,w \nvDash \psi$. Applying 
        induction hypothesis, we get : $M^f, [w] \nvDash \psi$. But then, we have $M^f, [w] \Vdash \phi$. Right to left is the same. \newline
        
        Case $\phi = \phi_1 \land \phi_2$: Suppose $\phi_1, \phi_2$ holds. Let $M,w \Vdash \phi$. That means $M,w \Vdash \phi_1$ and $M,w \Vdash \phi_2$. Applying
        induction hypothesis, we get $M^f,[w] \Vdash \phi_1$ and $M^f,[w] \Vdash \phi_2$. But then, $M^f,[w] \Vdash \phi_1 \land \phi_2 = \phi$. Right to left is similar. \newline
        
        Case $\phi = \Box_i\psi$ ($i \in \{1,2,\epsilon\}, \Box_\epsilon = \Box$): Left to right. Suppose $\psi$ holds and  $M,w \Vdash \Box_i \psi$. 
        We need to show $M^f,[w] \Vdash \Box_i \psi$, this means $\forall [v] \in W_\Sigma : [w]R_i[v] \rightarrow M^f,[v] \Vdash \Box_i \psi$. Pick any $[v] \in W_\Sigma$ s.t 
        $[w]R_i[v]$. By condition 3, w.r.t to the modal operator, we have $M,v \Vdash \psi$. By induction hypothesis, we get $M^f,[v] \Vdash \psi$. Because [v] was arbitrary it follows that $M^f,[w] \Vdash \Box_i \psi$. \newline \newline
        Right to left. Suppose $\psi$ holds and $M^f,[w] \Vdash \Box_i \psi$. Pick $v \in W$ s.t $wR_iv$. By condition 2, w.r.t to the modal operator, we have
        $[w]R^f_i[v]$. So, $M^f, [v] \Vdash \psi$. By induction hypothesis, we get $M,v \Vdash \psi$. Because v was arbitrary, we have $M,w \Vdash \Box_i \psi$.
    
    \end{proof}     
\end{theorem}


Now define the smallest filter for $L_{\Box,\Box_1,\Box_2}$ and show that this is a filter. We denote this as $R^s$.

\begin{definition}
    Let $M = (W,R,V)$ be a model, $\Sigma$ is closed under subformulas and $W_\Sigma$ the set of equivalence classes. We define :
    $$R^s = \{[w],[v] \mid \exists w' \in [w], \exists v' \in [v] : w'R_i v'\}$$
    where $i \in \{1,2,\epsilon\}$.    
\end{definition}

\begin{lemma}
    Let $M = (W,R,V)$ be a model, $\Sigma$ is closed under subformlas and $W_\Sigma$ the set of equivalence classes induced by $\equiv$ and $V^f$ the standard valuation on $W_\Sigma$. Then $(W_\Sigma, R^s,R^s_1,R^s_2,V^f)$ is 
    a filtration of M through $\Sigma$.
    \begin{proof}
    
    It suffices to show $R^s_i$ fullfills the condition 2 and 3 w.r.t to the corresponding modal operator $\Box_i$. But $R^s_i$ already satisfies condition 2.
    Let's check the other condition. Let $\Box_i \phi \in \Sigma$ , $[w]R^s_i[v]$ and $M,w \Vdash \Box_i \phi$ where $i \in \{1,2, \epsilon\}$. Because of $[w]R^s_i[v]$ we pick a $w' \in [w]$
    and $v' \in [v]$. By defintion, we have $w'R_i v'$. Because $w' \equiv w$, we get $M,w' \Vdash \Box_i \phi$. Hence, $M,v' \Vdash \phi$ and by $v' \equiv v$, we get $M,v \Vdash \phi$.
    
    \end{proof}    
\end{lemma}

\begin{proposition}
    Let $\Sigma$ be a finite subformula closed set of $L_{\Box,\Box_1,\Box_2}$. For any model M, if $M^f$ is a filtration through $\Sigma$, then $M^f$ contains at most $2^n$ nodes (where n denotes the size of $\Sigma$).

    \begin{proof}
    The states of $M^f$ are the equivalence classes in $W_\Sigma$. Let $g : W_\Sigma \rightarrow P(\Sigma)$ defined by $g([w]) = \{ \phi \in \Sigma \mid M,w \Vdash \phi\}$.
    g is well defined. Pick any u and v s.t $u\equiv v$. But then by defintion of $\equiv$, they fullfill the same subformulas. This means g([v]) = g([u]). \newline
    g is also injective. Pick any $[u],[v] \in W_\Sigma$ s.t g([u]) = g([v]). We show $[u]\subseteq[v]$. The other inclusion is similar.
    By assumption we have $u \equiv v$. Pick any $u' \in [u]$. Then we have $u' \equiv u \equiv v$. Hence, $u' \in [v]$. At the end, this means $M^f$ contains at most $2^n$ nodes.

    \end{proof}     
\end{proposition}


\begin{theorem}
    Let $\phi$ be a formula of $L_{\Box,\Box_1,\Box_2}$. If $\phi$ is satisfiable, then it is satisfiable on a finite model containing at most $2^n$ nodes, where n is the number of subformulas in $\phi$.

    \begin{proof}
    Assume that $\phi$ is satisfiable on a model on M. Take any filtration of M through the set of subformulas of $\phi$. 
    By Filtration Theorem, we get that $\phi$ is satisfied in the filtration model $M^f$. Furthermore, it is bounded by $2^n$.
    
    \end{proof}
    
    Now we define Sahlqvist formulas for our purposes.
        
\end{theorem}

\begin{definition}
    A modal formula $\phi$ is positive if all variables occurs in the scope of an even number of negations. In the other hand, a formula is negative, if all variables occurs in the scope of an odd number of negations.
    A boxed atom is a modal formula of the form $\Box^n p$ for some $n \in \mathbb{N}$, where p is a propositional variable and $\Box^n p$ is defined as follows :
    $\Box^0 p = p$, $\Box^1 p = \Box p$, $\Box^{n+1} p = \Box(\Box^n p)$. \newline \newline
    Furthermore, a Sahlqvist antecedent is built from $\bot, \top$, negative formulas and boxed atoms by applying $\Diamond$ and $\land$. A Sahlqvist implication is a modal formula of the form
    $\phi \rightarrow \psi$, where $\phi$ is a Sahlqvist antecedent and $\psi$ a positive formula. \newline
    Now, a Sahlqvist formula is built from Sahlqvist implications by applying $\Box$ and $\lor$. \newline
    
    Examples for Sahlqvist formulas: 
    $$\Box \Box p \rightarrow \Box p$$
    $$ \Diamond \neg p\rightarrow p$$
    $$ \Diamond \Box \Box \Box \Box \Box p\rightarrow \Box \Diamond \Box \Diamond p$$
    $$ \Box\Box\Box\Box(\Diamond \Box p\rightarrow p) \lor \Box \Box p \rightarrow \Box p $$
    
    Non Sahlqvist Formulas :
    $$\Box \Diamond p \rightarrow \Diamond \Box p$$
    $$\Diamond \Box p \rightarrow \Box \neg p$$ \newline
    
    We can extend this defintion for our logic. We say a boxed atom can be $\Box^n_1 p$ and $\Box^n_2 p$. A Sahlqvist antecedent
    can also be build by applying $\Diamond_1$ and $\Diamond_2$. A Sahlqvist formula can be build by
    Sahlqvist implications by applying additionally $\Box_1$ and $\Box_2$.  \newline
    
    Sahlqvist formulas possess important properties, which are guaranteed by the Sahlqvist Theorem. It says that, when given a normal modal logic K and a set of Sahlqvist formulas, the resulting logic is complete w.r.t 
    to the class of frames, which satisfies the corresponding first-order formula of the Sahlqvist formulas. This also holds for multimodal logic. We will not prove it here, but we will use this
    for our logic to show completeness w.r.t to a suitable class of frames.
    \newline
        
\end{definition}

\begin{definition}
    We say for a modal logic $\Lambda$ has the finite model property (FMP) if for every formula $\phi$ that is not provable in $\Lambda$,
    is falsifiable in a finite model.
        
\end{definition}

\begin{proposition}
    The logic $D \otimes D \otimes D + \Box p \rightarrow \Box_1 p \land \Box_2 p$ has FMP.
    \begin{proof}
    The idea is to pick a class of frame $\mathcal{F}$, where DNL is sound and complete with respect to and then show by filtration the FMP. \newline \newline
    Let $\mathcal{C} = \{F \mid F \Vdash DNL\}$. Obviously,  DNL is sound w.r.t $\mathcal{C}$. Completeness can be shown by using the Sahlqvist Theorem.
    We remember D = K + $\Box_i p \rightarrow \Diamond_i p$ for $i \in \{1,2, \epsilon\}$. These axioms and $\Box p \rightarrow \Box_1 p \land \Box_2 p$ are Sahlqvist formulas. 
    By Sahlqvist, we have that DNL is complete w.r.t $\{F \mid F \Vdash \forall x \exists y \, R_i(x,y) \mbox{ and } F \Vdash \forall x \forall y (R_1(x,y) \lor R_2(x,y)) \rightarrow R(x,y)\}$. \newline \newline
    Now assume a formula $\phi$ is not derivable from DNL. By completeness we get $\phi$ is falsifiable in a model M and a world w. Hence, $M,w \Vdash \neg \phi$. Now we build the set subf($\neg \phi$) which denotes the closed subformulas set of $\neg \phi$.
    Let $M^s = (W_\Sigma, R^s, R^s_1, R^s_2, V^s)$ be the smallest filtration of M through subf($\neg \phi$) and $V^s$ is the standard valuation on $W_\Sigma$. By filtration Theorem, they preserve truth.
    It remains to show $F^s = (W_\Sigma, R^s, R^s_1, R^s_2)\in  \mathcal{C}$. For that, every point must have at least one successor in every relation and it must hold that $R^s_1,R^s_2 \subseteq R^s$. For the first one, we show this for R because the rest is similar.
    Pick $[w]\in W_\Sigma$. Because M is based on a frame $F \in \mathcal{C}$, there is a point v s.t wRv. By the defintion of smallest filtration, we have that $[w]R^s[v]$. For the second one, we show only for $R^s_1$ because it is the same for $R^s_2$.
    Pick [w],[v] s.t $[w]R^s_1[v]$. By defintion of smallest filtration there are points $w' \in [w]$ and $v' \in [v]$ s.t $w'R_1v'$. Furthermore, it holds $R_1 \subseteq R$, so $w'Rv'$.
    By smallest filtration, we get $[w']R^s[v']$. Because $w' \equiv w \mbox{ and } v' \equiv v$, we have $[w'] = [w] \mbox{ and } [v'] = [v]$. It follows $[w]R^s[v]$.
    
    \end{proof}
        
\end{proposition}

\subsection{Completeness w.r.t $T_{\omega,\omega,\omega[rn]}$}

\begin{definition}
    Let $T_{\omega [in]}$ (r = reflexive, n = non-transitiv) denote the infinite branching and infinite depth tree, which is reflexive and non-transitive.
    Formally the tree can be defined as : $T_{\omega [in]} = (W, R)$ where $W = \mathbb{N}$* and sRt iff $\exists u \in \mathbb{N} \cup \{\epsilon\} : s*u = t$ (the '*' is the concatenation operator) \newline \newline
    The $T_{\omega,\omega,\omega [rn]}$ tree has three relations with infinite branching and infinite depth and we have $R_1,R_2 \subseteq R$. Before characterizing it, we say
    $\mathbb{N}_{1}$* is the set of finite number combinations which has a subscript $"1"$ to denote that these numbers relate to $R_1$  \newline 
    (examples : $2_{1}, 4231123_{1}$, $32_{1} \epsilon 45_{1} \epsilon 9_{1} = 32459_{1}$). \newline \newline
    Now let $T_{\omega,\omega,\omega [in]} = (W, R, R_1, R_2 )$ where $W = (\mathbb{N} \cup \mathbb{N}_{1} \cup \mathbb{N}_{2})^*$, 
    $$sRt \mbox{ iff } \exists u \in \mathbb{N} \cup \mathbb{N}_{1} \cup \mathbb{N}_{2} \cup \{\epsilon\} : s*u = t$$
    $$sR_1t \mbox { iff } \exists u \in \mathbb{N}_{1} \cup \{\epsilon\} : s * u = t$$
    $$sR_2t \mbox { iff } \exists u \in \mathbb{N}_{2} \cup \{\epsilon\} : s * u = t$$
    where $s,t \in W$. Again, the '*' operator acts here as a concatenation operator.
        
\end{definition}

\begin{definition}
    Let $F = (W,R_1, R_2, ...)$ and $F' = (W', R'_1, R'_2, ...)$ be two frames. A bounded morphism from $F \mbox{ to } F'$ is a function
    $f : W \rightarrow W'$ satisfying the following conitions: 
    
    $$ \mbox{ If } (u,v) \in R_i  \mbox{ then }(f(u), f(v)) \in R'_i $$
    $$ \mbox{ If } (f(w), v') \in R'_i \mbox{ then } \exists v \in W \mbox{ s.t } (w,v) \in R'_i \mbox{ and } f(v) = v'$$
    We say that $F'$ is a bounded morphic image of F, if there is a surjective bounded morphism from $F \mbox{ to } F'$.
        
\end{definition}

\begin{proposition}
    Let $\phi$ be a formula in $L_{\Box, \Box_1, \Box_2}$ , $F = (W, R, R_1, R_2)$ and $F' = (W', R', R'_1, R'_2)$ be two frames and $F \mbox{ to } F'$ a surjective bounded morphism. Then the following holds :
    $$ \mbox{ If }F \Vdash \phi \mbox{ then } F' \Vdash \phi$$
    This can be shown by structural induction on the length of the formula.
        
\end{proposition}

\begin{corollary}
    If $F'$ is a bounded morphic image of $F$, then we have $Log(F) \subseteq Log(F')$
    
\end{corollary}

\begin{proposition}
    D is sound and complete w.r.t $T_{\omega[in]}$.

    \begin{proof}
    Sound is clear. For completeness, we use the well known fact that D has FMP. This means, $D = Log\{F \mid F \Vdash D\}$ where $F$ is a finite frame.
    We can pick such a finite frame $F$ and it suffices to find a surjective bounded morphism f from $T_w$ to $F$. This would imply $Log(T_w) \subseteq D$. \newline
    Now, let $F = (W', R')$ be such a finite rooted frame with root w.(We can pick such because $F,w \Vdash D$ and then generate a subframe by w). We define 
    inductively an assignment of nodes of $F$ to the nodes of $T_\omega$. For the base case, we assign w to the root of $T_\omega$. The induction step looks like the following : 
    Assume a point $x \in T_\omega$ has been assigned to a point $u \in F$ but the successors of x has no assignment. 
    Let $s_1, s_2, ..., s_k$ be successors of u (k denotes amount of successors and $k\geq 1$ because seriality guarantees us at least one successor).
    For $n \geq 1, n \in \mathbb{N}$ we assign $s_i$ to the ($n * i$)th-successor of x. This means we are assigining the successors alternatingly. \newline
    Now we check for f the conditions of bounded morphism. First condition : Let x,y $\in T_\omega$ s.t xRy and f(x) = s. But then, y will be assigned to a successor point of s. 
    Hence, $f(x) R'f(y)$. Second condition : Suppose $f(x)R't$ and $f(x) = s$. Since t is a successor of s and f(x) = s, then a successor of x, say y, gets the assignment t. 
    
    \end{proof}
        
\end{proposition}

\begin{proposition}
    DNL is sound and complete w.r.t $T_{\omega,\omega,\omega[in]}$.

    \begin{proof}
    For soundness, we have that $T_{\omega,\omega,\omega[in]} \Vdash \Box p \rightarrow \Box_1 p \land \Box_2p$, because by defintion we have $R_1,R_2 \subseteq R$. The rest is clear.
    For completeness, we use the fact that DNL has FMP. Let $F = (W',R', R'_1, R'_2)$ be a finite rooted frame with root w and $F \Vdash DNL$. We define inductively an
    assignment similar to 2.17. We assign w to the root of $T_{\omega,\omega,\omega[in]}$. For induction step we start by only assigining points from $R_1$ to $R'_1$ and $R_2$ to $R'_2$. After that, the remaining points will be assigned to $R'$.
    The procedure works similar as described previously. \newline
    Now we check the conditions for $R_1 \mbox{ and } R'_1$. Let $xR_1y$ and $f(x) = s$. We assigned y to a successor of s. So, $f(x) R'_1f(y)$. For the second condition, the same argument holds as before.
    It works the same for $R_2 \mbox{ and } R'_2$. For $R$ and $R'$ we pick any $xRy$ with $f(x) = s$. 
    If we also have $xR_1y$, then it follows $f(x)Rf(y)$, because we showed that condtion for $R_1 \mbox{ and } R'_1$. The same argument holds if $xR_2y$. Else, the successor of s was assigned to y, so $f(x)R'f(y)$. 
    Let $f(x)R't$ and $f(x) = s$. If we have $f(x) R'_1 t$  and $x$ a point in $R_1$, then the second condition follows. The same holds for $f(x)R'_2t$ and $x$ in $R_2$.
    Else, we have that t is a successor of f(x) = s, and a successor of x was assigned to t.
    \end{proof}
        
\end{proposition}




\clearpage

\section{Topological Space Defintion}

\begin{definition}
    A topological space is a pair $(X, \tau)$ where $\tau$ is a collection of subsets of X (elements of $\tau$ are also called open sets) such that : 
    \newline
    \newline
    1. the empty set $\emptyset $ and X are open
    \newline
    2. the union of an arbitrary collection of open sets is open
    \newline
    3. the intersection of finite collection of open sets is open
    \newline
    \newline
    The space is called Alexandroff, if we allow the intersection of infinite collection of open sets.
    A topological model is a structure M = (X,$\tau$,v) where (X,$\tau$) is a topological space
    and v is a valuation assigining subsets of X to propositional variables. 
        
\end{definition}


\begin{definition}
    Let M = (X,$\tau$,v) a topological model and $x \in X$. The satisfaction of a formula
    at the point x in M is defined inductively as follows :
    \newline
    $M,x \models \Box \phi$ iff $\, \exists U \in \tau$ s.t $x \in U$ and $\forall u \in U : M,u \models \phi$
    \newline
    $M,x \models \Diamond \phi$ iff $\, \forall U \in \tau$ s.t $x \in U$ and $\exists u \in U : M,u \models \phi$
        
\end{definition}

\begin{definition}
    Let $A = (X, \chi)$ and B =(Y, $\upsilon$) be topological spaces. The standard product topology $\tau$ is the set of subsets of 
    $X \times Y$ such that $X \in \chi$ and $Y \in \upsilon$. \newline
    Let $N \subseteq X \times Y $. We call $N$ horizontally open if $\forall (x,y) \in N $ $\exists U \in \chi : x \in U $ and $ U \times \{ y \} \subseteq N$. \newline We call $N$ 
    vertically open if $\forall (x,y) \in N$ $\exists V \in \upsilon : y \in V$ and  $ \{ x \} \times V \subseteq N$ \newline
    If $N$ is H-open and V-open, then we call it HV-open. \newline
    We denote $\tau_1$ is the set of all H-open subsets of $X \times Y$ and $\tau_2$ is the set of all V-open subsets of $X\times Y$
        
\end{definition}


\begin{definition}
    Let X and Y be topological spaces and $f : X \rightarrow Y$ a function.
    We call f continuous if for each open set $U \subseteq Y$ the set $f^{-1}(U)$ is open in X. We say f 
    is open if for each open set $V \subseteq X$ the set f[V] is open in Y.
        
\end{definition}

\begin{remark}
    There is an alternative defintion for open sets. Let (X,$\tau$) be a topological space and U a set.
    U is open iff $\forall x \in U$  $\exists V\subseteq U$ : V is open and $x \in V$. This is true because,
    the union of open sets is an open set.
    \newline
    
    
\end{remark}
    
Now we define some Kripke frames, which we will use through this chapter.

\begin{definition}
    Let $T_2$ be the infinite binary tree with reflexive and transitive descendant relation. \newline Formally it is defined as follows :
    $T_2 = (W,R)$ where W = \{0,1\}* and sRt iff $\exists u\in W : s*u = t$. \newline
    The $T_ {6,2,2}$ tree is the infinite six branching tree, where all nodes of $T_ {6,2,2}$ is R-related, the first two R1-related 
    and the last two R2-related. Formally we can define this tree as follows : $T_ {6,2,2} = (W,R,R_1,R_2)$,
    where W = \{{0,1,2,3,4,5}\}*, $$sRt \mbox{ iff } \exists u \in \mbox{\{0,1,2,3,4,5\}*} : s*t = u $$
    $$ sR_1t  \mbox{ iff } \exists u \in \mbox{\{0,1\}*} : s*t = u$$
    $$ sR_2t  \mbox{ iff } \exists u \in \mbox{\{5,6\}*} : s*t = u$$
    
    where s and t are elements of the set where the element u can come from, w.r.t to the relation. For example in the case sRt,
    s and t are elements of \{0,1,2,3,4,5\}*.        
\end{definition}


\clearpage
\section{Neighborhood}
\subsection{Neighborhood frames}

\begin{definition}
    Let $X$ be a non-empty set. A function  $\tau : X \rightarrow 2^{2^X}$ is called a neighbourhood function. A pair 
    F = (X,$\tau$) is called a neighbourhood frame (or n-frame). A model based on F is a tuple (X,$\tau$,v), where v assigns a subset of X to a variable
        
\end{definition}

\vspace{0.5cm}

\begin{definition}
    Let $M$ =($X$,$\tau$,v) be a neighourhood model and $x$ $\in$ $X$. The truth of a formula is defined inductively as follows :
    $$M,x \models \Box \phi \mbox{ iff } \exists V \in N(x) \forall y \in V : M,y \models \phi$$ 
    A formula is valid in a n-model M if it is valid at all points of $M$ ($M \models \phi$). Formula is valid in a n-frame $F$ if it is valid in
    all models based on $F$ (notation $F \models \phi$). For Logic L we write $ F \models L, \mbox{ if for any }\phi \in L, F \models \phi$. 
    $\mbox{We define nV(L) =  } \{ F \mid F \mbox{ is an n-frame and } F \models \phi \}$.
\end{definition}

\begin{definition}
    Let $X$ be a non-empty set and $\tau$ neighborhood function. We call $\tau$ is a filter if for each $x\in X$ the collection $\tau(x)$
    satisfies the following conditions : \newline \newline
    1. $\emptyset \notin \tau(x)$ \newline
    2. $\mbox{If }U \in \tau(x)$ and $U \subseteq V$ then $V \in \tau(x)$ (upward closed) \newline
    3. $\mbox{If }U, V \in \tau(x)$, then $U \cap V \in \tau(x)$
\end{definition}


\begin{definition}
    Let F = (W,R) be a Kripke frame. We define an n-frame $N$(F) = (W, $\tau$) as follows.
    For any $w\in W$ we have :
    $$\tau(w) = \{ U \mid R(w) \subseteq U \subseteq W \}$$
        
\end{definition}

\vspace{0.5cm}
\begin{lemma}

    Let $F = (W,R)$ be a Kripke frame. Then $$Log(F) = Log(N(F))$$ 
    The proof is by structural induction.
\end{lemma}

\vspace{0.5cm}

\begin{definition}
    Let $X$ = (X, $\tau_1$,...) and $Y$ = (Y, $\sigma_1$,...) be n-frames. Then the function f:
    $X \rightarrow Y$ is called bounded morphism if \newline \newline
    1. f is surjective \newline
    2. $\forall x\in X \, \forall U \in \tau_i(x) : f(U) \in \sigma_i (f(x))$ \newline
    3. $\forall x\in X \, \forall V \in \sigma_i (f(x)) \, \exists U \in \tau_i(x) \, : f(U) \subseteq V$        
\end{definition}

\begin{corollary}
     Let $X$ = (X, $\tau_1$,...) and $Y$ = (Y, $\sigma_1$,...) be n-frames and $f : X \rightarrow Y$ a bounded morphism. Then 
     $$Log(X) \subseteq Log(Y)$$. \newline
     The proof is by structural induction.
\end{corollary}

\vspace{0.5cm}

\begin{definition}
    Let $X$ = (X, $\tau_1$) and $Y$ = (Y, $\tau_2$) be two n-frames. Then the product of these two frames
    is an n-2-frame and is defined as follows : \newline
    
    $$ X \times_n Y = (\mbox{X} \times \mbox{Y}, \tau_1', \tau_2')$$   
    $$ \tau_1'(x,y) = \{ U \subseteq \mbox{X} \times \mbox{Y} \mid \exists V \in \tau_1(x) : V \times  \{ y \} \subseteq U \}$$
    $$ \tau_2'(x,y) = \{ U \subseteq \mbox{X} \times \mbox{Y} \mid \exists V \in \tau_2(y) : \{ x \} \times V \subseteq U \}$$
    Additionally, we say the full product of n-frames $X \times^+_n Y$ is :
    $$ X \times^+_n Y = (X \times Y, \tau'_1, \tau'_2, \tau)\mbox{ where }$$
    $$ \tau(x,y) = \{ U \subseteq \mbox{X} \times \mbox{Y} \mid \exists W \in \tau_1(x) \, \exists V \in \tau_2(y) : W \times V \subseteq U \}$$        
\end{definition}

\begin{definition}
    For two unimodal logics $L_1$ and $L_2$ we define the n-product of them as follows :
    $$ L_1 \times_n L_2 = Log(\{ X \times Y \mid X \in nV(L_1) \mbox{ and } Y \in nV(L_2) \})$$        
\end{definition}



\subsection{Main Construction}
In the following, we will construct a useful neighborhood frame called $N_\omega[F]$ based on a frame F.
We will use it later, to show $L_1 \times_n L_2 = L_1 \otimes L_2$ where $L_1,L_2 \in \{D,S4,D4,T\}$

\begin{definition}
    Let $F = (A^*, R) = F_{\xi \eta}[A]$ and $0 \notin A$. We define "pseudo-infinite" sequences 
    $$X = \{a_1a_2a_3... \mid a_i \in A \cup \{0\} \mbox{ and } \exists N \forall k \geq N : a_k = 0\}$$
    Furthermore, we define $f_F : X \rightarrow A^*$ to be the function, that deletes all zeros. \newline \newline
    Example : Say $12034002340^\omega \in X$ ($0^\omega$ denotes infinitely many zeros). Then $f_F(12034002340^\omega) = 1234234$

\end{definition}

\begin{definition}
    Let $F = (A^*, R) = F_{\xi \eta}[A]$ and $0 \notin A$. Assume the function $f_F$ and the set $X$ as defined before.
    For $\alpha \in X$ such that $\alpha = a_1a_2...$ we define 
    
    \begin{align*}
            st(a) &= min\{N \mid \forall k \geq N : a_k = 0\} \\
            a \mid_{k} &= a_1a_2...a_k \\
            U_k(\alpha) &= \{ \beta \mid f_F(\alpha)Rf_F(\beta) \mbox{ and } \alpha \mid_m = \beta \mid_m,  \mbox{ where } m = max((k, st(\alpha))\} 
    \end{align*} 
    Remark : Let $\alpha \in X \mbox{ with } st(\alpha) = n$. Then we have that $U_n(\alpha) = U_j(\alpha)$ for any $j \leq n$

\end{definition}

\begin{lemma}
    $U_k(\alpha) \subseteq U_m(\alpha)$, whenever $k \geq m$.
    
    \begin{proof}
        Let $\beta \in U_k(\alpha)$. Since $\alpha \mid_k = \beta \mid_k$ and $k \geq m$, we have $\alpha \mid_m = \beta \mid_m$. It follows, $\beta \in U_m(\alpha)$.
    \end{proof}
\end{lemma}

\begin{definition}
    Due to Lemma 5.2.3 the sets $U_n(\alpha)$ forms a filter base. So we can define :
    $$\tau(\alpha) \mbox{ is a filter with base } \{U_n(\alpha) \mid n \in \mathbb{N} \}$$
    $$N_\omega = (X, \tau) \mbox{ is the n-frame based on } F$$

\end{definition}

\begin{lemma}
    Let $F = (A^*, R) = F_{\xi \eta}[A]$. Based on that, let $N_\omega(F) = (X,\tau)$, $N(F) = (A^*, \sigma)$ and $f_F : N\omega(F) \rightarrow N(F)$.
    Then for any $m \in \mathbb{N}$ and $x\in X$ with $x = a_1a_2...$ \newline we have $$f_F(U_m(x)) = R(f_F(x))$$
    \vspace{0.01cm}
    \begin{proof}
            $\subseteq$ : Let $f_F(\alpha) \in f_F(U_m(x))$ with $\alpha \in U_m(x)$. By defintion of $U_m(x)$,
            we get $f_F(\alpha) \in R(f_F(x))$. \newline \newline
            For the other direction, we pick $\vec{a} \in R(f_F(x))$. We have to find $\beta \in U_m(x)$ s.t $f_F(\beta) = \vec{a}$. We assume R is irreflexive and non-transitive.
            The other cases are similar. \newline
            Because $\vec{a} \in R(f_F(x))$, there must exists $c \in A$ such that $ \vec{a} = f_F(x) * c $.
            We construct $\beta = x \mid_m \cdot \, c\, 0^\omega$. Hence, $f(\beta) = f_F(x \mid_m) * f_F(c)$ and because $ 0 \notin A$ we get $f_F(x \mid_m) * c = \vec{a}$.
            
            
    \end{proof}
\end{lemma}

\begin{lemma}
    Let $F =(A^*,R) = F_{\xi \eta}[A]$. Then $f_F : N_\omega(F) \rightarrow N(F)$ is a bounded morphism.

    \begin{proof}
        From now on this proof we will omit the subindex in $f_F$.
        Let $N_\omega(F) = (X, \tau)$ and $N(F) =(A^*, \sigma)$. \newline For surjectivity, we pick any $\vec{x} \in A^*$. But then, $\vec{x}\,0^\omega \in X$. Hence, $f(\vec{x}\, 0^\omega) = \vec{x}$. \newline 
        For the next condition, assume that $x \in X \mbox{ and } U \in \tau(x)$. We need to prove that $f(U) \in \sigma(f(x))$. That means $R(f(x)) \subseteq f(U)$. Because $U \in \tau(x)$, there is a $m$ such that
        $U_m(x) \subseteq U$. By Lemma 5.2.5 we have $f(U_m(x)) = R(f(x))$. It follows, 
        $$R(f(x)) = f(U_m(x)) \subseteq f(U)$$ 
        \newline
        Assume $x\in X$ and $V$ is a neighborhood of x, i.e $R(f(x)) \subseteq V$. We need to prove that there exists $U \in \tau(x)$, such that $f(U)\subseteq V$.
        As $U$ we $U_m(x)$ for any $m \in \mathbb{N}$. By Lemma 5.2.5 we get $f(U_m(x)) = R(f(x))$. Hence, 
        $$f(U_m(x)) = R(f(x)) \subseteq V$$

    \end{proof} 
\end{lemma}

\begin{corollary}
    For frame $F = F_{\xi \eta}[A]$ we have $Log(N_\omega(F)) \subseteq Log(F)$.
    \begin{proof}
        It follows from Lemma 5.1.5, Corollary 5.1.7 and Lemma 5.2.6 
        $$Log(N_\omega(F)) \subseteq Log(N(F)) = Log(F)$$
    \end{proof}

\end{corollary}

\begin{proposition}
    Let $F_{in} = F_{in}[\mathbb{N}], F_{rn} = F_{rn}[\mathbb{N}], F_{it} = F_{it}[\mathbb{N}] \mbox{ and } F_{rt} = F_{rt}[\mathbb{N}]$. Then
    $$Log(N_\omega(F_{in})) = D$$
    $$Log(N_\omega(F_{rn})) = T$$
    $$Log(N_\omega(F_{it})) = D4$$
    $$Log(N_\omega(F_{rt})) = S4$$

    \begin{proof}
        In all these cases, the inclusion from left to right follows from Proposition 2.0.9 and Corollary 5.2.7.
        Now the converse direction. Assume $X =(X,\tau)$. \newline
        It is easy to check that $X \vDash D$ iff for each $x \in X : \emptyset \notin \tau(x)$. For $N_\omega(F_{in})$ and $N_\omega(F_{it})$ this holds. \newline
        It is easy to check that $X \vDash T$ iff we have $x\in U \in \tau(x)$ for any $x$ and $U$. For $N_\omega(F_{rn})$ and $N_\omega(F_{rt})$ this holds. \newline
        Now we check $X \vDash 4$ iff for each $U \in \tau(x) : \{y \mid U \in \tau(y)\} \in \tau(x)$. \newline
        $\supseteq : $ Let $x \in X$ and assume $X,x \vDash \Box p$. That means there exists $U \in \tau(x)$ s.t $U\subseteq V(p)$.
        By assumption we have $S = \{y \mid U \in \tau(y)\} \in \tau(x)$. But then $X,x \vDash \Box\Box p$ because we can just pick the set S. \newline
        $\subseteq :$ By contradiction, assume there exists a $U$ s.t $\{y \mid U \in \tau(y)\} \notin \tau(x)$. Let $X,x \vDash \Box p$ where $V(p) = U$.
        If $X,x \vDash \Box\Box p$ then it must be the case that  \newline 
        $S =\{y \in X \mid X,y \vDash \Box p\} \in \tau(x)$. $X,y \vDash \Box p$ means $U \in \tau(y)$. That means $S = \{y \in X \mid U \in \tau(y)\}$. But by assumption $S \notin \tau(x)$. 
        Hence, $X,x \not\vDash \Box\Box p$. But thats a contradiction. This also holds for $N_\omega(F_{it})$ and $N_\omega(F_{rt})$ because we have for any $y \in U_m(x)$ and 
        $k \geq m : U_k(y) \subseteq U_m(x)$.
            
    \end{proof}
\end{proposition}

\begin{definition}
    to be continued...
\end{definition}

\begin{lemma}
    Let $T_{\omega,\omega,\omega[rn]}$ be as in Definition 3.2.1. Then 
    $$Log(T_{\omega,\omega,\omega[rn]}) = Log(N(T_{\omega,\omega,\omega[rn]}))$$ \newline
    The proof is by structural induction.

    
\end{lemma}

    Assume $F_1 = (A^*,R_1) = F_{\eta_1 \xi_1}[A]$ and $F_2 = (B^*,R_2) = F_{\eta_2 \xi_2}[B]$ 
    with $A \cap B = \emptyset$, $A = \{a_1,a_2,a_3,...\}$ and $B = \{b_1,b_2,b_3,...\}$. 
    Consider the product of n-frames $F'_1 = (X_1, \tau_1) = N_\omega(F_1)$ and $F'_2 = (X_2, \tau_2) = N_\omega(F_2)$ is 
    $$X = (X_1 \times X_2, \tau'_1, \tau'_2) = N_\omega(F_1) \times_n N_\omega(F_2)$$ \newline
    Furthermore, we have $F_1 \otimes F_2 = ((A \cup B)^*, R_1', R_2')$ as defined in Defintion 2.0.7.
    We consider the neighborhood version $$N(F_1 \otimes F_2) = ((A \cup B)^*, \sigma_1', \sigma_2')$$ \newline
    Now we define $g : X_1 \times X_2 \rightarrow (A \cup B)^*$ as follows. For $(\alpha, \beta) \in X_1 \times X_2$ with $\alpha = a_1a_2...$ and $\beta = b_1b_2...$
    we define $g(\alpha,\beta)$ to be the finite sequence which we get after eliminating all zeros from the infinite sequence
    $a_1b_1a_2b_2...$ \newline \newline
    Example : Let $\alpha = 012340^\omega$  and $\beta = 0ab00e0^\omega$. Then $g(\alpha, \beta) = 1a2b34e$.

\begin{lemma}
    Let $X$ and $N(F_1 \otimes F_2)$ be as defined before and $(\alpha,\beta) \in X_1 \times X_2$. Then for any $m >$ max$\{st(\alpha), st(\beta)\}$ we have 
    $$R_1'(g(\alpha,\beta)) = g(U_m(\alpha) \times \{\beta\})$$.

    \begin{proof}
        We will show both direction by assuming $F_1 = F_{in}[A]$. For $F_{it}[A],F_{rt}[A]$ and $F_{rn}[A]$ is it similar. \newline
        $\subseteq$ : Let $\vec{w} \in R'_1(g(\alpha,\beta))$. By definition we get, there exists $\vec{c} \in A^*$ where $\vec{w} = g(\alpha,\beta) \cdot \vec{c}$ and $\Lambda R_1 \vec{c}$.
        Because $R_1$ is irreflexive and non-transitive, we get $\vec{c} \in A$. We construct $(\zeta,\beta)$ where $\zeta \in U_m(\alpha)$ and $g(\zeta,\beta) = \vec{w}$. For that, we can take $\zeta = \alpha \mid_m \cdot \vec{c}\, 0^\omega$. Obviously, $\zeta \in U_m(\alpha)$.
        Because $m \in max\{st(\alpha), st(\beta)\}$, we have $g(\alpha \mid_m, \beta \mid_m) = g(\alpha, \beta)$. Hence, $g(\zeta, \beta) = g(\alpha \mid_m, \beta \mid_m) \cdot g(\vec{c} \,0^\omega, 0^\omega) = g(\alpha \mid_m, \beta \mid_m) \cdot \vec{c} = \vec{w}$.
        \newline \newline
        $\supseteq$ : Assume $\zeta \in U_m(\alpha)$. We have to show $g(\zeta, \beta) \in R_1'(g(\alpha, \beta))$. Because $R_1$ is irreflexive and non-transitive, it suffices to find a $\vec{c} \in A$ s.t. $g(\alpha,\beta) \cdot \vec{c} = g(\zeta, \beta)$.
        By choosing $m$ is maximal and $\zeta \mid_m = \alpha \mid_m$, we have $g(\zeta_m, \beta_m) = g(\alpha,\beta)$. We also know $f_F(\alpha)R_1 f_F(\zeta)$, that means there exists a $\vec{d} \in A$ s.t. $f_F(\alpha) \cdot \vec{c} = f_F(\zeta)$. This $\vec{d}$ must appear at a point after $\zeta_m$.
        We can follow $g(\zeta, \beta) = g(\alpha, \beta) \cdot \vec{d}$. Hence, $g(\zeta, \beta) \in R_1'(g(\alpha,\beta))$.
        \newline \newline
        Remark : If $m \leq max\{st(\alpha), st(\beta)\}$, then we cannot gurantee the equality. Assume $R_1$ as before and $\alpha = 123000^\omega$ and $\beta = d0b0a0^\omega$.
        Lets pick $m = 3$ and $\zeta = 123100^\omega$. Then we have $g(\alpha, \beta) = 1d23ba$ and $g(\zeta, \beta) = 1d23b1a$. Obviously, we don't have $g(\zeta, \beta) \in R_1'(g(\alpha,\beta))$. \newline
        Furthermore, we can show $R'_2(g(\alpha,\beta)) = g(\alpha \times U_m(\beta))$ similar as above. 
        
    \end{proof}


\end{lemma}

\begin{lemma}
    Function $g : X \rightarrow N(F_1 \otimes F_2)$ is a bounded morphism.

    \begin{proof}
        Let $\vec{z} = z_1z_2...z_n \in (A \cup B)^*$. Define for $i \leq n$ :
        \[
            x_i = 
            \begin{cases}
            z_i, & \text{if } z_i \in A; \\
            0, & \text{if } z_i \notin A.
            \end{cases}
            \qquad
            y_i = 
            \begin{cases}
            z_i, & \text{if } z_i \in B; \\
            0, & \text{if } z_i \notin B.
            \end{cases}
        \] \newline
        Let $\alpha = x_1x_2...x_n \, 0^\omega$ and $\beta = y_1y_2...y_n \, 0^\omega$. Then $g(\alpha,\beta) = \vec{z}$. Hence, g is surjective. \newline \newline
        For the next conditions we check only for $\tau_1'$ and $\sigma_1$. The other case is similar.
        Assume $(\alpha,\beta) \in X_1 \times X_2$ and $U \in \tau_1'(\alpha, \beta)$. We have to show $g(U) \in \sigma_1(g(\alpha,\beta))$. That means $R_1'(g(\alpha,\beta)) \subseteq g(U)$.
        Pick a $m > max\{st(\alpha), st(\beta)\}$ s.t. $U_m(\alpha) \times \{\beta\} \subseteq U$. We can pick such m, because $U \in \tau_1'(\alpha,\beta)$. So there exists $U_k(\alpha) \in \tau(\alpha)$ s.t. $U_k(\alpha) \times \{\beta\} = U$.
        If $k > max\{st(\alpha), st(\beta)\}$ then we are done. Else, by Lemma 5.2.3, we have for any $n \geq k : U_n(\alpha) \subseteq U_k(\alpha)$. So we can lift the $k$ til we reach $m$. Then we use Lemma 5.2.11 to get the following :
        $$R_1'(g(\alpha,\beta)) = g(U_m(\alpha) \times \{\beta\}) \subseteq g(U)$$
        \newline
        For the last condition we assume $(\alpha, \beta) \in X_1 \times X_2$ and $V \in \sigma_1(g(\alpha,\beta))$ (or rather $R_1'(g(\alpha,\beta)) \subseteq V$).
        We need to prove there exists $U \in \tau_1'(\alpha,\beta)$, such that $g(U) \subseteq V$. As $U$ we take $U_m(\alpha) \times \{\beta\}$ for some $m > max\{st(\alpha), st(\beta)\}$. Hence, by Lemma 5.2.11 we get :
        $$g(U_m(\alpha) \times \{\beta\}) = R_1'(g(\alpha,\beta)) \subseteq V$$ 
    \end{proof}
    
\end{lemma} 

    Now we will show $g : N_\omega(T_{\omega[rn]}) \times^+_n N_\omega(T_{\omega[rn]}) \rightarrow N(T_{\omega,\omega,\omega[rn]})$ is a bounded morphism. \newline 
    Let $(T_{\omega[rn]})_1 = (\mathbb{N}_{1}, R_1)$ and $(T_{\omega[rn]})_2 = (\mathbb{N}_{2}, R_2)$ as defined in Defintion 3.2.1.
    Then we take the n-frames $N_\omega(T_{\omega[rn]})_1 = (X_1, \tau_1)$ and $N_\omega(T_{\omega[rn]})_2 = (X_2, \tau_2)$. For the proof, we will omit the subscripts after the frame. \newline
    Let's consider the full product of the n-frames :
    $$N_\omega(T_{\omega[rn]}) \times^+_n N_\omega(T_{\omega[rn]}) = (X_1 \times X_2, \tau'_1, \tau'_2, \tau)$$
    \newline We say $N(T_{\omega,\omega,\omega[rn]}) = ((\mathbb{N}_1 \cup \mathbb{N}_2\cup \mathbb{N})^*, \sigma_1, \sigma_2, \sigma)$ where the tree is as defined in Def.3.2.1\newline \newline
    In order to define the bounded morphism, we have to fix a bijection first. Let $h : \mathbb{N}_1 \times \mathbb{N}_2 \rightarrow \mathbb{N}$ be a bijection. \newline
    Next, we define function $k :(\mathbb{N}_1 \cup \{0\}) \times (\mathbb{N}_2 \cup \{0\}) \rightarrow \mathbb{N}_1 \cup \mathbb{N}_2 \cup \mathbb{N} \cup \{0\}$ as follows :
    \[
        f(a, b) =
        \begin{cases}
        a, & \text{if } b = 0; \\
        b, & \text{if } a = 0; \\
        h(a, b), & \text{otherwise}.
        \end{cases}
    \]

    Let $(\alpha,\beta) \in X_1 \times X_2$ with $\alpha = a_1a_2...$ and $\beta = b_1b_2...$. We define 
    $$g'(\alpha,\beta) = f(a_1,b_1)f(a_2,b_2)...$$
    At last, we define $g(\alpha,\beta)$ as $g'(\alpha,\beta)$ but removing all zeros.
    \clearpage
    \begin{lemma}
        Function $g : N_\omega(T_{\omega[rn]}) \times^+_n N_\omega(T_{\omega[rn]}) \rightarrow N(T_{\omega,\omega,\omega[rn]})$ is a bounded morphism.
        
        \begin{proof}
            Let $\vec{z} = z_1z_2...z_n \in (\mathbb{N}_{1} \cup \mathbb{N}_{2} \cup \mathbb{N})^*$. We define for $i \leq n $
        \[
            x_i = 
            \begin{cases}
            z_i, & \text{if } z_i \in \mathbb{N}_{1}; \\
            a_1, & \text{if } z_i = h(a_1,b_2); \\
            0,   & \text{other}
            \end{cases}
            \qquad
            y_i = 
            \begin{cases}
            z_i, & \text{if } z_i \in \mathbb{N}_{2}; \\
            b_2, & \text{if } z_i = h(a_1,b_2); \\
            0,   & \text{other } 
            \end{cases}
        \] 
        \newline
        Let $\alpha = x_1x_2...x_n \, 0^\omega$ and $\alpha = y_1y_2...y_n \, 0^\omega$. But then $g(\alpha,\beta) = \vec{z}$. Hence, g is surjective.
        For the next conditions we do it for $\tau$ and $\sigma$. \newline
        First we need to show, $R(g(\alpha,\beta)) = g(U_m(\alpha) \times U_m(\beta))$ for $m > max\{st(\alpha), st(\beta)\}$.
        The proof is similar as in Lemma 5.2.1. \newline
        $\subseteq$ : Pick $\zeta \in U_m(\alpha), \upsilon \in U_m(\beta)$. Show $g(\zeta, \upsilon) \in R(g(\alpha,\beta))$. By definition, we need to find a $\vec{c} \in \mathbb{N}_1 \cup \mathbb{N}_2 \cup \mathbb{N} \cup \{\epsilon\}$ s.t. $g(\alpha,\beta) \cdot \vec{c} = g(\zeta, \upsilon)$.




        \end{proof}
    \end{lemma}



    



\end{document}

